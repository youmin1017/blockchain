\section{Environment Setup}

本專題我們使用Next.js作為我們WEB3 Restaurant的前端與後端Framework,並使用Hardhat
工具開發相關的智能合約,以下詳細說明我們所使用到的工具。

\subsection{Frontend Dependencies}

\begin{itemize}
  \item @walletconnect/web3wallet: The Web3Modal SDK allows us to easily
  connect your Web3 app with wallets.
  \item @web3modal/ethers: For interacting with Ethereum smart contracts.
  \item tailwindcss: A utility-first CSS framework for rapidly building custom
  designs.
  \item shadcn-ui: React components
\end{itemize}

\subsection{Blockchain Technologies}

\begin{itemize}
  \item Pinata: 用於管理我們在IPFS(星際檔案系統)上的檔案
  \item Hardhat: Ethereum development environmen
  \item OpenZepplin: Smart contract Library
\end{itemize}

\section{Program Description}

本專題主要包含兩個部分,分別為智能合約與網頁前後端。首先是智能合約部分,我們的餐聽
系統有兩個合約,分別為NFT優惠券合約與餐廳管理合約,網頁前後端的部分擇使用Next.js框架
實作,Figure~\ref{fig:architecture}為本專案的架構圖。

\begin{figure}
  \centering
  \includegraphics*[width=0.75\textwidth]{architecture.png}
  \caption{Project Architecture}
  \label{fig:architecture}
\end{figure}

\subsection{Restaurant Contracts}

\subsubsection{Restaurant NFT Coupon Contract With ERC1155}

本次專題利用ERC1155製作了擁有餘額概念的NFT,相比於ERC721每個Token都是獨一無二的,
使用ERC1155製作優惠券更加符合我們的需求,使用ERC1155我們就可以讓我們就可以一次
鑄造多張優惠券,再將優惠券分別分給不同的客戶。此外我們還利用Pinata將我們的優惠
券圖片與相關的Metadata上傳至IPFS中,在前端我們就可以透過讀取Metadata將這我們製
作的優惠券顯示在前端中。上傳成功後的結果如Figure~\ref{fig:ipfs}所示,此外我們的
NFT也能夠在OpenSea的測試網中看到,如Figure~\ref{fig:opensea}所示,能夠看到我們
優惠券的樣式與剩餘數量。


\begin{figure}[H]
  \centering
  \includegraphics*[width=0.75\textwidth]{ipfs.png}
  \caption{IPFS}
  \label{fig:ipfs}
\end{figure}

\begin{figure}[H]
  \centering
  \includegraphics*[width=0.75\textwidth]{opensea.png}
  \caption{OpenSea}
  \label{fig:opensea}
\end{figure}

\begin{quote}
``Coupon NFT Address: Oxb561B7A33B0ef3b7fa6117C39F940E656d0516ab``
\end{quote}


\subsubsection{Restaurant Management Contract}

餐廳管理合約主要負責管理餐廳的資訊,包括座位管理、點餐系統、優惠券功能。在使用
優惠券訂餐時,我們會先檢查使用者的優惠券數量,若數量足夠,則會呼叫Coupon合約中的
useCoupon函數,來使用優惠券。此外本次專題中我們實作的合約都有利用到OpenZepplin的
AccessControl,來管理合約的權限,讓我們的合約更加安全。

\subsection{Next.js Frontend}

本專題我們使用Next.js作為我們的前端框架,該網頁能夠使用WalletConnect來連接錢包,
登入後如果為管理員可以透過優惠券管理頁面來管理優惠券,如發送優惠券給使用者,如
Figure~\ref{fig:coupon-manage}所示。一般客人則可以透過網頁來點餐,使用優惠券,
如Figure~\ref{fig:demo}所示。

\begin{figure}[H]
  \centering
  \includegraphics*[width=0.75\textwidth]{coupon-manage.png}
  \caption{Coupon Management}
  \label{fig:coupon-manage}
\end{figure}

\begin{figure}[H]
  \centering
  \includegraphics*[width=0.75\textwidth]{demo.png}
  \caption{Demo}
  \label{fig:demo}
\end{figure}

\section{The Difficulty We Faced and Solution}

本次專題中,我們原先想延續先前的作業,使用ERC721來製作優惠券,但後來發現ERC721
所製作的NFT每個Token都是獨一無二的,這樣的設計令我一張優惠券只能發送給一個使用
者,非常的不方便,好在後來在OpenZepplin的官方文檔中找到了ERC1155,這個標準可以
讓我們的NFT有餘額的概念,這樣我們就可以一次鑄造多張優惠券,再將優惠券分給不同的
客戶,這樣的設計更加符合我們的需求。然而因為時間上的關係,整體網頁還有許多進步
的空間,例如新增Loading畫面、新增NFT優惠券,這些都是我們未來可以進一步改進的地
方。

\section{Summary}

本次專題我們實作了一個WEB3 Restaurant,使用ERC1155製作了擁有餘額概念的NFT優惠券,
並使用Next.js作為我們的前端框架,透過WalletConnect來連接錢包,讓使用者可以透過錢包
來登入,並使用優惠券來點餐。此外我們也使用了Hardhat來開發智能合約,並使用Pinata
來管理我們在IPFS上的檔案,讓我們的NFT優惠券能夠在前端中顯示。整體來說,本次專題
讓我們學習到了許多WEB3的相關技術,也讓我們更加熟悉智能合約的開發。
