\section{Environment Setup}

For using hardhat we should install Node.js LTS(v20)\cite{nodejs} and hardhat's
dependencies. We choose fnm\cite{fnm} to install Node.js, fnm is Node.js
manager which makes installing specific version of Node.js easily, and
yarn\cite{yarn} to install hardhat hardhat's dependencies. For using Hardhat Ignition
to deploy our smart contract we should install followings dependencies

\begin{itemize}
  \item @nomicfoundation/hardhat-ignition-ethers
  \item @nomicfoundation/hardhat-ethers
  \item @nomicfoundation/hardhat-ignition
  \item @nomicfoundation/hardhat-verify
  \item @nomicfoundation/ignition-core
\end{itemize}

\section{Program Description}

Benefit from OpenZeppelin ERC-20 library, if we want to make our custom token we only
need to inherit it in our contract and optionaly specify name, decimal and
symbol. In this homework, hence, we write new bank smart contract to manage our
token. The bank contract offers some baisc mehods for client, deposit, withdraw,
transfer, mintFromTYToken, etc, and offers shutdown(not showen in
Figure~\ref{fig:clayoat}) method for contract owner to frezze the contract.

\subsection{Bank Faucet Methods}

This section dscripbes the methods about our ERC-20 token contract and bank contract.

\subsubsection{ERC-20}

\begin{itemize}
  \item approve: Sets amount as the allowance of spender over the caller’s tokens.
  \item allowance: Returns the remaining number of tokens that spender will be
        allowed to spend on behalf of owner through transferFrom
  \item transfer: Transfer token to other address.
  \item transferFrom: Transfer token from address to another address.
  \item mint: the public method to allow anyone mint tokens.
  \item balanceOf: Get balance of address.
  \item totalSupply: Get the number totaly minted token.
\end{itemize}

\subsubsection{Token Bank}

In this section we just dscribe the necessary methods that bank required.

\begin{itemize}
  \item deposit: Transfer the token to the bank contract and record it to each
  account's balance.
  \item withdraw: Withdraw from contract, will check balance is enough first.
  \item transfer: Transfer deposited token to another address.
  \item mintFromToken: This method allow anyone to mint the token into bank.
  \item shutdown(owner only): Freez the contract.
\end{itemize}

\begin{figure}[H]
    \centering
    \includegraphics*[width=\textwidth]{tyt-bank.png}
    \caption{TYToken Bank System Design}
    \label{fig:clayout}
\end{figure}

\section{Hardhat Testing Contracts on Local Network}
\label{sec:hardhat-testing}

After finishing our contracts, we write unit testing for the contracts. Belows
showswhat functionalities we are tested. The result showen in
Figure~\ref{fig:testing}

\begin{itemize}
  \item TYToken
  \begin{itemize}
    \item Mint Token
    \item Transfer token to other account
    \item Approve and TransferFrom from account to another
    \item Balance checking
    \item Allowance checking
  \end{itemize}
  \item Bank
  \begin{itemize}
    \item Mint from TYToken contract
    \item Deposit token into bank
    \item Withdraw token from bank
    \item Transfer token to other account
  \end{itemize}
\end{itemize}

\begin{figure}[H]
    \centering
    \includegraphics*[width=\textwidth]{testing.png}
    \caption{Contracts Unit Testing}
    \label{fig:testing}
\end{figure}

\section{Deployment}

\subsection{Local Network Deployment}

In this homework, we use hardhat-ignition to deploy our contract. Firstly, we
use ignition to deploy TYToken in local network to testing. The result showen
in Figure~\ref{fig:ignition}. Then, to deploy our contracts we only need to
specify the network to Sepolia and configure Infura api key in hardhat config.
However, when we trying to deploy our bank contract in Sepolia we got
ERC20InvalidReceiver error, showen in Figure~\ref{fig:erc20-error}, we do not
know why this error happening currently. In future, we will keep trying to
solve this error.

\begin{quote}
``Hardhat Ignition is a declarative deployment system that enables you to
deploy your smart contracts without navigating the mechanics of the deployment
process.``
\end{quote}
 
\begin{figure}[h]
    \centering
    \includegraphics*[width=\textwidth]{ignition.png}
    \caption{tytoken ignition deployment}
    \label{fig:ignition}
\end{figure}

\begin{figure}[h]
    \centering
    \includegraphics*[width=\textwidth]{ERC20InvalidReceiver.png}
    \caption{ERC20InvalidReceiver Error}
    \label{fig:erc20-error}
\end{figure}

\subsection{Deploy Contracts in Sepolia using Remix IDE}

Due to the ERC20InvalidReceiver error, in this homework, we still using Remix
IDE to deploy and test our contracts in Sepolia network. The result showen in
Figure~\ref{fig:remix-deploy}. After the contracts deployed to the Sepolia, we
could import our token into MetaMask to check how much token we have. The
result showen in Figure~\ref{fig:metamask-tyt}.

\begin{figure}[H]
    \centering
    \includegraphics*[width=\textwidth]{Remix-Deploy.png}
    \caption{Remix IDE Deployment}
    \label{fig:remix-deploy}
\end{figure}

\begin{figure}[H]
    \centering
    \includegraphics*[width=0.5\textwidth]{Metamask-TYT.png}
    \caption{MetaMask TYT Token}
    \label{fig:metamask-tyt}
\end{figure}

\subsection{Deployment Flow}

Hardhat ignition offers visualize method to visualize our deployment flow. If
project becmoe more and more complex, we could simply use this tool to overview
our project quickly. The result showen in Figure~\ref{fig:deploy-flow}.

\begin{figure}[H]
    \centering
    \includegraphics*[width=0.55\textwidth]{deployment-flow.png}
    \caption{Deployment Flow}
    \label{fig:deploy-flow}
\end{figure}


\section{The Difficulty We Faced and Solution}

原先我們預計在本次計畫中設計一個簡單的前端介面來測試我們新增的Token,但礙於資料
與經驗的不足、時間因素,這個想法暫時被我們擱置著。此外,也碰到了在測試網路中部
署成功與但在Sepolia中部署失敗的情況,同樣也因為時間的因素,暫時還未解決。

\section{Summary}

這次的作業我們製作了屬於自己的Token、基於前次作業的經驗透過Bank合約管理我們製作
的Token,雖然過程中碰到了不少問題,但也透過本次作業更加地了解到了以太坊設計的厲
害之處。

