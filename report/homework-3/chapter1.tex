\section{Environment Setup}

For using hardhat we should install Node.js LTS(v20)\cite{nodejs} and hardhat's
dependencies. We choose fnm\cite{fnm} to install Node.js, fnm is Node.js manager, and
yarn\cite{yarn} to install hardhat hardhat's dependencies.

\section{Program Description}

這次的作業我們完善了作業一的銀行系統,將當中的Faucet合約分離成為一個獨立的合約、透過Mapping紀錄客戶資訊
、新增OpenZeppelin權限管理等功能改善。銀行合約架構如Figure~\ref{fig:clayout}所示,Bank與Faucet合約繼承自
Freezable合約,使二者擁有關機的能力,在發現合約有誤時,合約擁有者能夠將合約凍結。

\begin{figure}[H]
    \centering
    \includegraphics*[width=\textwidth]{contract-layout.png}
    \caption{銀行合約架構}
    \label{fig:clayout}
\end{figure}

\subsection{Bank Contract Description}

銀行合約提供了註冊、註銷、提款、轉帳等銀行所需的基本功能,如Figure~\ref{fig:clayout}
所示,該合約透過OpenZeppelin/AccessControl新增了兩個Role,分別為Owner(Manager)與Client
,接著再透過OpenZeppelin/AccessControl中的onlyRole這個Modifier來限制不同Methods的權限
管理。以下介紹該合約各Methods的功能。

\begin{itemize}
  \item Register:註冊成為銀行客戶(管理者限定)
  \item Deregister:註銷銀行帳戶,且退回所有幣(管理者限定)
  \item IsRegistered:確認是否為銀行客戶
  \item Deposit:存款至銀行中(客戶限定)
  \item Withdraw:從銀行提款(客戶限定)
  \item Transfer:轉帳給其他客戶(客戶限定)
  \item GetBalance:確認餘額(客戶限定)
  \item WithdrawFromFaucet:從Faucet合約中提款至銀行中(客戶限定)
  \item Shutdown:凍結合約,使所有Methods失效並將所有款項轉給擁有者(管理者限定)
\end{itemize}

\subsection{Faucet Contract Description}

Faucet合約非常簡單,只提供了提款與關機功能,該合約也同樣有使用到OpenZeppelin/AccessControl
來進行權限管理,只有合約擁有者能夠將合約凍結。

\section{Hardhat Testing Contracts on Local Network}
\label{sec:hardhat-testing}

因區塊鏈不可更改的特性,合約的正確性就變得非常重要,為了確保合約是否能夠正確部
署、合約中各個Methods的正確性,我們利用Hardhat提供的單元測試,分別為Faucet與
Bank合約撰寫了各功能的單元測試。主要測試項目如下,結果如Figure~\ref{fig:testing}
所示。

\begin{itemize}
  \item Faucet
  \begin{itemize}
    \item Deployment
    \item Owner is correct
    \item Able to withdraw
    \item Cannot withdraw more than 1 ether
  \end{itemize}
  \item Bank
  \begin{itemize}
    \item Deployment
    \item Owner is correct
    \item Able to register
    \item Able to withdraw from Faucet
    \item Able to withdraw from Bank
    \item Able to deposit to Bank
    \item Able to transfer to other client
    \item Able to deregister
  \end{itemize}
\end{itemize}

\begin{figure}[H]
    \centering
    \includegraphics*[width=\textwidth]{testing.png}
    \caption{Hardhat單元測試}
    \label{fig:testing}
\end{figure}

註:該次測試僅測試了主要功能是否能夠正常運作,實際專案中需要更詳盡的測試各種狀
況,以避免以太幣被盜走的情況。

\subsection{Hardhat Gas Reporter}

 在利用Hardhat測試的過程中,我們利用Hardhat Gas Reporter來預估各個Methods的Gas
 消耗量,如Figure~\ref{fig:gas-reporter}所示,以Register這個Method為例,最小、
 最大、平均的Gas花費分別為78,605、81,405、79,005。而當我們實際部署到Sepolia
 Network時Register的實際Gas花費為81,361 Figure~\ref{fig:register-gas}。此工具是
 透過CoinMarketCap提供的API計算相關的Gas花費,因此可以得到測試與實際上相近的結
 果,未來在開發智能合約時也能夠透過此工具來檢測各Method的Gas花費狀況。
 
\begin{figure}[H]
    \centering
    \includegraphics*[width=\textwidth]{hardhat-gas-reporter.png}
    \caption{Hardhat Gas Reporter}
    \label{fig:gas-reporter}
\end{figure}

\begin{figure}[H]
    \centering
    \includegraphics*[width=\textwidth]{register-gas.png}
    \caption{Register Gas Consume}
    \label{fig:register-gas}
\end{figure}



\section{Deployment to Sepolia Network}

本次專案我們同樣透過Remix部署Faucet與Bank合約,並在Remix中進行以下測試,詳細測
試紀錄如Figure~\ref{fig:etherscan}所示。

\begin{itemize}
  \item Faucet測試
  \begin{enumerate}
    \item Able to withdraw
  \end{enumerate}
  \item Bank測試
  \begin{enumerate}
    \item 未註冊提款,結果如Figure~\ref{fig:withdraw-fail}
    \item Able to register
    \item Able to withdraw from faucet(0x62dfa61d)
    \item Able to transfer
    \item Able to deposit
    \item Able to withdraw
  \end{enumerate}
\end{itemize}

得益於先前Section~\ref{sec:hardhat-testing}中的單元測試,合約中的程式邏輯錯誤都
在測試過程中解決掉了,因此合約部署上Sepolia後功能也都得以正常執行。

\begin{figure}[H]
  \centering
  \includegraphics*[width=\textwidth]{withdraw-fail.png}
  \caption{因未註冊提款失敗}
  \label{fig:withdraw-fail}
\end{figure}


\begin{figure}[H]
  \centering
  \includegraphics*[width=\textwidth]{etherscan.png}
  \caption{Etherscan紀錄}
  \label{fig:etherscan}
\end{figure}

\begin{figure}[H]
  \centering
  \includegraphics*[width=\textwidth]{events.png}
  \caption{Withdraw Event}
  \label{fig:withdraw-event}
\end{figure}

\section{The Difficulty We Faced and Solution}

關於這次作業,對我們來說最大的困難點是區塊鏈因為較新的緣故,網路上的資源相對較
少、文檔也相對不完整,尤其是OpenZeppelin關於AccessControl這方面,很多程式碼都是
片段,也較少完整的範例讓我們理解,這點對於背景知識不足的我們來說是相對吃力的。
所幸後續也在不斷地嘗試、討論中解決了問題。

\section{Summary}

透過這次作業我們學習到了如何整合不同的合約、OpenZeppelin/AccessControl、Hardhat
工具的使用、智能合約的單元測試與部署,讓我們更加地了解以太坊的生態,也相信這些
工具能對我們的期末專題有很大的幫助。

