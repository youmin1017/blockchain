\section{Environment Setup}

For using hardhat we should install Node.js LTS(v20)\cite{nodejs} and hardhat's
dependencies. We choose fnm\cite{fnm} to install Node.js, fnm is Node.js
manager which makes installing specific version of Node.js easily, and
yarn\cite{yarn} to install hardhat hardhat's dependencies. Moreover, in this
homework we use yarn's workspace feature to manage our project, so we only need
to run yarn install in the project root directory to install all dependencies.

\subsection{Frontend Dependencies}

\begin{itemize}
  \item @walletconnect/web3wallet: The Web3Modal SDK allows us to easily
  connect your Web3 app with wallets.
  \item @web3modal/ethers: For interacting with Ethereum smart contracts.
  \item tailwindcss: A utility-first CSS framework for rapidly building custom
  designs.
  \item shadcn-ui: React components
\end{itemize}

\section{Program Description}

本次作業因使用到了Chainlink中的VRF功能,我們必須先透過Chainlink提供的Faucet取得
測試用的LINK,並新增VRF的Subscription,取得Subscription ID。接著我們設計了一個
智能合約,透過Chainlink VRF產生一個隨機數字,並將該數字與20取餘數,回傳一個在預
先定義好的Houses Array中取得房名。最後我們透過Next.js實作了一個前端網頁,讓使用
者能夠透過網頁與智能合約互動。

\subsection{VRF Random House Generator Contract}

\subsubsection{VRF Random House Generator Methods}

\begin{itemize}
  \item \textbf{constructor}: Initialize the contract with subscription id
  which is used to subscribe to Chainlink VRF. 
  \item \textbf{rollDice}: Request a random number from Chainlink VRF.
  \item \textbf{house}: Return the house name of given address.
  \item \textbf{fulfillRandomWords}: Callback function for Chainlink VRF.
  \item ... (other not important methods)
\end{itemize}

\begin{figure}[H]
    \centering
    \includegraphics*[width=0.55\textwidth]{remix-contract.png}
    \caption{WalletConnect}
    \label{fig:remix-contract}
\end{figure}


\subsubsection{具體實作流程如下}

\begin{enumerate}
  \item 首先,在Chainlink提供的Faucet中取得測試用的LINK。
  \item 接著在Chainlink的VRF中註冊一個Subscription,取得Subscription ID。
  \item 部署合約時,將Subscription ID作為參數傳入。
  \item 合約部署完成後,在Chainlink中,將該合約的地址註冊為Consumer。
  \item 使用者呼叫rollDice函數,合約會向Chainlink VRF發送請求。
  \item Chainlink VRF回傳隨機數字後,合約會將該數字與20取餘數,並將結果儲存。
  \item 使用者呼叫house函數,合約會回傳該使用者的房名。
\end{enumerate}

\subsection{Next.js Frontend}

本作業中,我們使用Next.js實作了一個簡單的前端網頁,讓使用者能夠連接Metamask、
WalletConnect等錢包Figure~\ref{fig:frontend-connect},並透過網頁與智能合約互動。
使用者可以透過網頁呼叫rollDice函數,並透過呼叫house函數取得自己的房名。如
Figure~\ref{fig:demo}所示右上角為已經連接到Metamask錢包,並可透過Disconnect按鈕
斷開連接;左下角三個按鈕分別為Get Owner、Roll Dice、Get House,分別對應到智能合
約中的owner、rollDice、house函數;右下角則為按下Get House後成功呼叫合約中的house
method,取得隨機骰到的房子名稱。

\begin{figure}[H]
    \centering
    \includegraphics*[width=0.50\textwidth]{frontend-connect.png}
    \caption{WalletConnect}
    \label{fig:frontend-connect}
\end{figure}


\begin{figure}[H]
    \centering
    \includegraphics*[width=0.8\textwidth]{frontend-house.jpeg}
    \caption{WalletConnect}
    \label{fig:demo}
\end{figure}


\section{The Difficulty We Faced and Solution}

在本次作業中,收先,我們遇到最大的困難是在今年的四月29日(2024/04/29),VRF剛剛
更新為2.5版本Figure~\ref{fig:vrf2.5-news},而導致跟以往的作業相比,網路上的資訊
又更少了,甚至以英文作為關鍵字的資訊也是少之又少,所幸Chainlink的官方文檔還算完
整,讓我們成功完成了基於VRF2.5隨機產生房名的智能合約。接著我們遇到最大的問題是
如何在前端連接錢包、與智能合約互動,這部分我們參考了WalletConnect的官方文檔,該
文檔提供了相當高品質的範例程式碼,讓我們成功地完成了前端的部分。

\begin{figure}[H]
    \centering
    \includegraphics*[width=\textwidth]{chainlink-vrf2.5-news.png}
    \caption{VRF2.5 news}
    \label{fig:vrf2.5-news}
\end{figure}

\section{Summary}

這次作業我們成功地利用Chainlink取得了一個Verifiable的Random Number,也理解到了
在區塊鏈的世界中,該如何取得一個真正的隨機數字。此外,我們也成功地利用Next.js實
作了一個前端網頁,讓使用者能夠透過網頁與智能合約互動。這次作業讓我們更加熟悉了
智能合約的開發流程,為我們的期末專期打下了良好的基礎。

